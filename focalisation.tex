\documentclass{article}
\usepackage[utf8]{inputenc}
\usepackage[british]{babel}
\usepackage{csquotes}
\usepackage[margin=0.8in]{geometry}
\usepackage{microtype}
\usepackage{amsmath}
\usepackage{amssymb}
\usepackage{amsthm}
\usepackage{cmll}
\usepackage{stmaryrd}
\usepackage{ebproof}
\usepackage[
    style=alphabetic,
    sorting=none
]{biblatex}
\addbibresource{focalisation.bib}
\DeclareLabelalphaTemplate{
    \labelelement{
        \field[ifnames=1,strwidth=10]{labelname}
        \field[ifnames=2-,strwidth=1]{labelname}
    }
    \labelelement{
        \field[strwidth=2,strside=right]{year}
    }
}
\usepackage{xcolor}
\definecolor{linkcolor}{HTML}{DD00BB}
\usepackage[
    pdfauthor={Naïm Favier},
    bookmarksopen=true,
    colorlinks=true,
    allcolors=linkcolor
]{hyperref}
\usepackage{cleveref}

\author{Naïm Favier}

\setlength{\parskip}{0.5em}
\renewcommand\labelitemi{$\circ$}

\newtheorem{theorem}{Theorem}
\newtheorem{corollary}{Corollary}[theorem]
\newtheorem{lemma}[theorem]{Lemma}

\newcommand\LL{\textsf{LL}}
\newcommand\LLfoc{{\LL_\text{foc}}}
\newcommand\LLFoc{{\LL_\text{Foc}}}
\newcommand\size[1]{{\lvert #1 \rvert}}
\newcommand\sem[1]{{\llbracket #1 \rrbracket}}
\newcommand\biperp{{\perp\perp}}
\newcommand\triperp{{\perp\perp\perp}}
\newcommand\Foc{\text{Foc}}

\begin{document}

\section{A proof of focalisation}

Let $\vdash \Gamma; \Pi$ denote a sequent of $\LLfoc$ as defined in~\cite{laurent}.

We define the focalised syntactic phase model as $(M, \bot, \varphi)$ where $M$ is the free commutative monoid over formulas of MALL, $\bot = \{\Gamma \in M \mid\,\vdash \Gamma;\}$, and $\varphi(X) = \{X\}^\perp$ for positive atoms $X$. Let $\sem{A}$ be the interpretation of a formula $A$ in this model.

For a formula $A$, let $\size{A}$ denote the number of main negative subformulas in $A$. Define $\Psi_A$ as an $\size{A}$-ary monotonous operator on $\mathcal P(M)$ by induction:
\begin{itemize}
    \item $\Psi_N(N_1) = N_1$ if $N$ is negative
    \item $\Psi_X() = \{X^\perp\}$
    \item $\Psi_{B \otimes C}(B_1, \dots, B_\size{B}, C_1, \dots, C_\size{C}) = \Psi_B(B_1, \dots, B_\size{B}) \cdot \Psi_C(C_1, \dots, C_\size{C})$
    \item $\Psi_{B \oplus C}(B_1, \dots, B_\size{B}, C_1, \dots, C_\size{C}) = \Psi_B(B_1, \dots, B_\size{B}) \cup \Psi_C(C_1, \dots, C_\size{C})$
    \item $\Psi_1() = \{\emptyset\}$
    \item $\Psi_0() = \emptyset$
\end{itemize}

\begin{lemma}
    \label{positive_phase}
    For any formula A with main negative subformulas $A_1, \dots, A_\size{A}$, $\Psi_A(\{A_1\}^\perp, \dots, \{A_\size{A}\}^\perp)^\biperp \subseteq \{A\}^\perp$.
\end{lemma}
\begin{proof}
    To simplify the notations, let $\vdash \Gamma; N$ mean $\vdash \Gamma, N;$ when $N$ is a negative formula.
    Let $\Foc(A) = {\{\Gamma \in M \mid\, \vdash \Gamma; A\}}$. Clearly $\Foc(A) \subseteq \{A\}^\perp$ by the \textit{foc} rule.

    We prove by induction on $A$ that $\Psi_A(\Foc(A_1), \dots, \Foc(A_\size{A})) \subseteq \Foc(A)$:
    \begin{itemize}
        \item If $A$ is negative, the result is trivial.
        \item If $A = X$, then $\Psi_X() = \{X^\perp\} \subseteq \Foc(X)$ by the \textit{ax} rule.
        \item If $A = B \otimes C$, then we have \begin{align*}
            \Psi_A(\Foc(A_1), \dots, \Foc(A_\size{A})) &= \Psi_B(\Foc(B_1), \dots, \Foc(B_\size{B})) \cdot \Psi_C(\Foc(C_1), \dots, \Foc(C_\size{C})) \\
            &\subseteq \Foc(B) \cdot \Foc(C)
        \end{align*} by the induction hypothesis; moreover,
        $$\begin{prooftree}
            \hypo{\vdash \Gamma; B}
            \hypo{\vdash \Delta; C}
            \infer2[$\otimes$]{\vdash \Gamma, \Delta; B \otimes C}
        \end{prooftree}$$
        hence $\Foc(B) \cdot \Foc(C) \subseteq \Foc(B \otimes C)$, from which the result follows.
        \item If $A = B \oplus C$, then we have \begin{align*}
            \Psi_A(\Foc(A_1), \dots, \Foc(A_\size{A})) &= \Psi_B(\Foc(B_1), \dots, \Foc(B_\size{B})) \cup \Psi_C(\Foc(C_1), \dots, \Foc(C_\size{C})) \\
            &\subseteq \Foc(B) \cup \Foc(C)
        \end{align*} by the induction hypothesis; moreover,
        $$\begin{prooftree}
            \hypo{\vdash \Gamma; B}
            \infer1[$\oplus_1$]{\vdash \Gamma; B \oplus C}
        \end{prooftree}
        \quad
        \begin{prooftree}
            \hypo{\vdash \Delta; C}
            \infer1[$\oplus_2$]{\vdash \Delta; B \oplus C}
        \end{prooftree}$$
        hence $\Foc(B) \cup \Foc(C) \subseteq \Foc(B \oplus C)$, from which the result follows.
        \item If $A = 1$, clearly $\Psi_1() = \{\emptyset\} \subseteq \Foc(1)$ by the $1$ rule.
        \item If $A = 0$, clearly $\Psi_0() = \emptyset \subseteq \Foc(0)$.
    \end{itemize}

    Since $A_1, \dots, A_\size{A}$ are negative, we have \begin{align*}
        \Psi_A(\{A_1\}^\perp, \dots, \{A_\size{A}\}^\perp)^\biperp &= \Psi_A(\Foc(A_1), \dots, \Foc(A_\size{A}))^\biperp \\
        &\subseteq \Foc(A)^\biperp \\
        &\subseteq \{A\}^\triperp = \{A\}^\perp
    \end{align*}
\end{proof}

\begin{lemma}
    \label{positivity}
    For any formula A with main negative subformulas $A_1, \dots, A_\size{A}$, $\sem{A} = \Psi_A(\sem{A_1}, \dots, \sem{A_\size{A}})^\biperp$.
\end{lemma}
\begin{proof}
    By induction, using positivity results from~\cite[appendix F]{girard}: $(X^\biperp \cdot Y^\biperp)^\biperp = (X \cdot Y)^\biperp$ and $(X^\biperp \cup Y^\biperp)^\biperp = (X \cup Y)^\biperp$.
    \begin{itemize}
        \item If $A$ is negative, then $\sem{A} = \sem{A}^\biperp$ because $\sem{A}$ is a fact.
        \item If $A = X$, let $\Gamma \in \{X\}^\perp$ and $\Delta \in \{X^\perp\}^\perp$. We have
        $$\begin{prooftree}
            \hypo{\vdash \Gamma, X;}
            \hypo{\vdash X^\perp, \Delta;}
            \infer2[\textit{n-cut}]{\vdash \Gamma, \Delta;}
        \end{prooftree}$$
        from which $\{X\}^\perp \subseteq \{X^\perp\}^\biperp$ follows; moreover $\{X^\perp\}^\biperp = \Psi_X()^\biperp \subseteq \{X\}^\perp$ by lemma~\ref{positive_phase}, therefore $\sem{X} = \{X\}^\perp = \{X^\perp\}^\biperp$.
        \item If $A = B \otimes C$, then \begin{align*}
            \sem{B \otimes C} &= (\sem{B} \cdot \sem{C})^\biperp \\
            &= (\Psi_B(\sem{B_1}, \dots, \sem{B_\size{B}})^\biperp \cdot \Psi_C(\sem{C_1}, \dots, \sem{C_\size{C}})^\biperp)^\biperp \\
            &= (\Psi_B(\sem{B_1}, \dots, \sem{B_\size{B}}) \cdot \Psi_C(\sem{C_1}, \dots, \sem{C_\size{C}}))^\biperp \\
            &= \Psi_{B \otimes C}(\sem{B_1}, \dots, \sem{B_\size{B}}, \sem{C_1}, \dots, \sem{C_\size{C}})^\biperp
        \end{align*}
        \item If $A = B \oplus C$, then \begin{align*}
            \sem{B \oplus C} &= (\sem{B} \cup \sem{C})^\biperp \\
            &= (\Psi_B(\sem{B_1}, \dots, \sem{B_\size{B}})^\biperp \cup \Psi_C(\sem{C_1}, \dots, \sem{C_\size{C}})^\biperp)^\biperp \\
            &= (\Psi_B(\sem{B_1}, \dots, \sem{B_\size{B}}) \cup \Psi_C(\sem{C_1}, \dots, \sem{C_\size{C}}))^\biperp \\
            &= \Psi_{B \oplus C}(\sem{B_1}, \dots, \sem{B_\size{B}}, \sem{C_1}, \dots, \sem{C_\size{C}})^\biperp
        \end{align*}
        \item If $A = 1$ then $\sem{1} = \{\emptyset\}^\biperp$ by definition.
        \item If $A = 0$ then $\sem{0} = \emptyset^\biperp$ by definition.
    \end{itemize}
\end{proof}

\begin{lemma}
    \label{sem_A_inc_A_perp}
    For any formula $A$, $\sem{A} \subseteq \{A\}^\perp$.
\end{lemma}
\begin{proof}
    By induction:
    \begin{itemize}
        \item If $A = X^\perp$, we have $\{X^\perp\} \subseteq \Foc(X) \subseteq \{X\}^\perp$, therefore $\sem{X^\perp} = \sem{X}^\perp = \{X\}^\biperp \subseteq \{X^\perp\}^\perp$.
        \item If $A = B \with C$, we have $\sem{B \with C} = \sem{B} \cap \sem{C} \subseteq \{B\}^\perp \cap \{C\}^\perp$ by the induction hypothesis; moreover,
        $$\begin{prooftree}
            \hypo{\vdash \Gamma, B;}
            \hypo{\vdash \Gamma, C;}
            \infer2[$\with$]{\vdash \Gamma, B \with C;}
        \end{prooftree}$$
        hence $\{B\}^\perp \cap \{C\}^\perp \subseteq \{B \with C\}^\perp$, from which the result follows.
        \item If $A = B \parr C$, let $\Gamma \in \sem{B \parr C} = (\sem{B}^\perp \cdot \sem{C}^\perp)^\perp$. By the induction hypothesis, $\sem{B} \subseteq \{B\}^\perp$, hence $B \in \{B\}^\biperp \subseteq \sem{B}^\perp$, and similarly $C \in \sem{C}^\perp$, therefore $\vdash B, C, \Gamma;$\;. Moreover,
        $$\begin{prooftree}
            \hypo{\vdash \Gamma, B, C;}
            \infer1[$\parr$]{\vdash \Gamma, B \parr C;}
        \end{prooftree}$$
        hence $\Gamma \in \{B \parr C\}^\perp$, therefore $\sem{B \parr C} \subseteq \{B \parr C\}^\perp$.
        \item If $A = \top$, we have $\sem{\top} = M = \{\top\}^\perp$ by the $\top$ rule.
        \item If $A = \bot$, we have $\sem{\bot} = \bot \subseteq \{\bot\}^\perp$ by the $\bot$ rule.
        \item Otherwise, $A$ is a positive formula with main negative subformulas $A_1, \dots, A_\size{A}$.
        Then, \begin{align*}
            \sem{A} &= \Psi_A(\sem{A_1}, \dots, \sem{A_\size{A}})^\biperp &&\text{by \cref{positivity}} \\
            &\subseteq \Psi_A(\{A_1\}^\perp, \dots, \{A_\size{A}\}^\perp)^\biperp &&\text{by the induction hypothesis and monotonicity of $\Psi_A$} \\
            &\subseteq \{A\}^\perp &&\text{by \cref{positive_phase}}
        \end{align*}
    \end{itemize}
\end{proof}

\begin{corollary}
    \label{sem_Gamma_inc_Gamma_perp}
    For any multiset of formulas $\Gamma = A_1, \dots, A_n$, $\sem{\Gamma} \subseteq \{\Gamma\}^\perp$.
\end{corollary}
\begin{proof}
    By~\cref{sem_A_inc_A_perp}, we have $\sem{A_i} \subseteq \{A_i\}^\perp$ for all $1 \le i \le n$, hence $\{A_i\} \subseteq \{A_i\}^\biperp \subseteq \sem{A_i}^\perp$, therefore $\{\Gamma\} = \{A_1\} \cdots \{A_n\} \subseteq \sem{A_1}^\perp \cdots \sem{A_n}^\perp$.

    Thus, $\sem{\Gamma} = \sem{A_1} \parr \dots \parr \sem{A_n} = (\sem{A_1}^\perp \cdots \sem{A_n}^\perp)^\perp \subseteq \{\Gamma\}^\perp$.
\end{proof}

\begin{theorem}[Focalised completeness]
    If a sequent $\vdash \Gamma$ of MALL is valid in all phase models, then $\vdash \Gamma$ has a focalised proof.
\end{theorem}
\begin{proof}
    In particular $\emptyset \in \sem{\Gamma}$, hence $\emptyset \in \{\Gamma\}^\perp$ by \cref{sem_Gamma_inc_Gamma_perp}, therefore there is a proof $\pi$ of $\vdash \Gamma;$ in $\LLfoc$. Then, using~\cite[section 3.2]{laurent}, we get a proof $\pi'$ of $\vdash \Gamma;$ in $\LLFoc$. Finally, by~\cite[proposition 2]{laurent}, $\pi'^\circ$ is a cut-free, focalised proof of $\vdash \Gamma$ in $\LL$.
\end{proof}

Combining this with the soundness theorem for phase models, we get:
\begin{corollary}[Focalisation]
    Every provable sequent $\vdash \Gamma$ of MALL has a focalised proof.
\end{corollary}

\printbibliography

\end{document}
